% !TEX TS-program = xelatex
% !TEX encoding = UTF-8 Unicode
\documentclass[gap.tex]{subfiles}
\begin{document}
\label{sec:speed-section-route}
Points are awarded for the time taken to complete a subset of the task. This
timed section we call the speed section. In the published results, a speed is
given to each pilot completing the speed section but what distance should we
use to calculate this speed?

The rules say:

"Speed section distance is defined as the path of shortest distance from start
of speed section to end of speed section that touches all turnpoint cylinders."

It is possible for the speed section distance $\overrightarrow{DE}$ to be much
shorter than the distance across the speed section $\overrightarrow{FG}$
following the optimal route for the task $\overrightarrow{HI}$, as show in
Fig~\ref{fig:speed-section-optimal-route}.

\begin{figure}[ht]
    \centering
    \begin{tikzpicture}
\tkzDefPoint(-2,-5){U}
\tkzDefPoint(-1,-5){V}
\tkzDefPoint(-2,-5){P}
\tkzDefPoint(1,-5){Q}
\tkzDrawCircle[ color=green
              , fill=green!40
              , opacity=.5
              ](P,Q)
\tkzDefPoint(1,1){R}
\tkzDefPoint(4,1){S}
\tkzDrawCircle[ color=red
              , fill=red!40
              , opacity=.5
              ](R,S)
\tkzDefPoint(-2,4){C}
\tkzDefPoint(-1,4){A}
\tkzDrawCircle[ color=blue
              , fill=blue!40
              , opacity=.5
              ](U,V)
\tkzDrawCircle[ color=blue
              , fill=blue!40
              , opacity=.5
              ](C,A)
\tkzInterLC(U,R)(P,Q) \tkzGetPoints{}{D}
\tkzInterLC(U,R)(R,S) \tkzGetPoints{E}{}
\tkzInterLC(U,C)(P,Q) \tkzGetPoints{}{F}
\tkzInterLC(U,C)(R,S) \tkzGetPoints{G}{}
\tkzInterLC(U,C)(C,A) \tkzGetPoints{I}{}
\tkzInterLC(U,C)(U,V) \tkzGetPoints{}{H}
\tkzDrawPoints(C,R,U,D,E,F,G,H,I)
\tkzDrawSegment(G,F)
\tkzDrawSegment(E,D)
\tkzDrawSegment[style=dotted](G,I)
\tkzDrawSegment[style=dotted](H,F)
\tkzLabelPoints[below left](D)
\tkzLabelPoints[above right](E,F,G,H,I)
\tkzLabelCircle[ draw
               , fill=yellow!20
               , text centered
               ](U,V)(180){launch}
\tkzLabelCircle[ draw
               , fill=yellow!20
               , text centered
               ](P,Q)(15){start speed}
\tkzLabelCircle[ draw
               , fill=yellow!20
               , text centered
               ](R,S)(15){end speed}
\tkzLabelCircle[ draw
               , fill=yellow!20
               , text centered
               ](C,A)(180){goal}
\end{tikzpicture}

    \caption{Task optimal route and speed section optimal route.}
    \label{fig:speed-section-optimal-route}
\end{figure}

\end{document}
