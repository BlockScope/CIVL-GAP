% !TEX TS-program = xelatex
% !TEX encoding = UTF-8 Unicode
\documentclass[gap.tex]{subfiles}
\begin{document}
CIVL’s scoring system is generally known as “GAP”, named after the first-name
initials of its three inventors Gerolf Heinrichs (G), Angelo Crapanzano (A) and
Paul Mollison (P). Their intention was to “create a fair scoring system easily
adaptable to any competition anywhere in the world, both for hang gliding and
paragliding, with a philosophy that is easy for the pilot to understand,
regardless of the mathematical complexity of the underlying formulas”.

\subsection{History}
Work on GAP started in 1998, and it was officially introduced in 2000, to allow
scoring of competitions based on GPS track logs, instead of photographic
evidence as it had been used until then.

In 2002, an updated version, named “GAP 2002” was published. This introduced
the concept of leading points, which are calculated by comparing the complete
track logs of all pilots in a task. Leading points replaced the departure
points used in GAP 2000

In 2005, a variation of GAP 2002 was introduced in Australia, named “OzGAP” or
“OzGAP 2005”. The difference to GAP lies mainly in the way arrival points are
calculated, but this was never adopted by CIVL.

In 2008, “GAP 2008” was officially released. The main scoring mechanisms
remained unchanged from the 2002 edition, but the implementation of GAP 2008
included several rules introduced in the sporting codes for either hang-gliding
or paragliding. These cover stopped tasks, starting too early, and landing
between the end of the speed section and goal.

In 2011, “GAP 2011” marked another software release where the main scoring
remained unchanged from the 2002 definition and implementation. The main
changes were all for paragliding: altitude bonus in stopped tasks, as well as
a reduced number of available points in stopped tasks and in tasks with no
pilots in goal.

In 2012, the “Jump the Gun” rule for early starts in hang-gliding competitions
changed in S7A. This was implemented in FS, but this was also released,
unfortunately, as “GAP 2011”.

The 2014 edition introduced a number of significant changes for paragliding,
a few of which also applied to hang gliding. The majority of those changes
originated from the Paragliding World Cup Association (PWCA), and 2014 was the
first time that both CIVL and PWCA scored their paragliding competitions using
the same formula. The changes were:
\begin{itemize}
    \item
        Nominal launch parameter, see~\ref{sec:nominal-launch} (hang gliding
        and paragliding)
    \item
        Final glide decelerator, see~\ref{sec:final-glide-decelerator}
        (paragliding)
    \item
        Goal shape, see~\ref{sec:goal-definition} (paragliding)
    \item
        Purely linear distance points, see~\ref{sec:distance-points}
        (paragliding)
    \item
        Adjusted formula for leading points, see~\ref{sec:leading-points}
        (paragliding)
    \item
        No more arrival points, see~\ref{sec:arrival-points} (paragliding)
    \item
        Scoring of stopped tasks, see~\ref{sec:stopped-tasks} (hang gliding and
        paragliding)
    \item
        Use of FTV for competition scoring,
        see~\ref{sec:overall-competition-ranking} (paragliding)
\end{itemize}
The 2015 edition applies the 2014 changes in leading points calculations for
paragliding now also to hang gliding. In paragliding, the use of final glide
deceleration methods is no longer mandatory. QNH is now used as the primary
altitude measurement. Distance calculations continue to be based on the FAI
sphere, the introduction of a new distance measurement regime, based on the
WGS84 ellipsoid, has been postponed.

2016: In Paragliding, the leading points weight (maximum available leading
points) is doubled compared to the previous version. This reduces the available
time points. Also, if no pilot is in goal, the weight is now calculated as the
ratio between task distance and actual distance covered by the pilot who flew
the furthest. The maximum in this case is 0.1 (equivalent to 100 points for
a task with task quality 1). In hang gliding, the penalty for jumping the gun
was increased for one point every three seconds to one point every 2 seconds.

\subsection{Scoring Process}
Scoring follows a nine-step process, as depicted in
Figure~\ref{fig:scoring-process}:

\begin{figure}[!ht]
    \centering
    \begin{tikzpicture}[node distance=2cm]

\node (start) [startstop] {Start};
\node (comp) [io, below of=start, xshift=3cm] {1: Define competition parameters};
\node (task) [io, below of=comp, xshift=-6cm] {2: Define task};
\node (fly) [process, below of=task] {3: Fly task};
\node (track) [decision, below of=fly] {4: Evaluate track logs};
\node (valid) [decision, below of=track] {5: Calculate task validity};
\node (points) [decision, below of=valid] {6: Allocate available points};
\node (score) [decision, below of=points] {7: Score flights};
\node (ranktask) [rank, below of=score] {8: Rank task};
\node (rankcomp) [rank, below of=ranktask] {9: Rank competition};
\node (stop) [startstop, below of=rankcomp] {Stop};

\draw [arrow] (start) -| (comp);
\draw [arrow] (start) -| (task);
\draw [arrow] (task) -- node[anchor=west] {task definition} (fly);
\draw [arrow] (fly) -- node[anchor=west] {track logs} (track);
\draw [arrow] (track) -- node[anchor=west] {for all pilots: distance, time to ESS} (valid);
\draw [arrow] (valid) -- node[anchor=west] {task validity} (points);
\draw [arrow] (points) -- node[anchor=west] {available points} (score);
\draw [arrow] (score) -- node[anchor=west] {pilots' scores} (ranktask);
\draw [arrow] (ranktask) -- node[anchor=west] {task results} (rankcomp);
\draw [arrow] (rankcomp) -- node[anchor=west] {competition results} (stop);

\draw [arrow] (comp) |- (valid);
\draw [arrow] (comp) |- (points);
\draw [arrow] (comp) |- (score);

\end{tikzpicture}

    \caption{Scoring process}
    \label{fig:scoring-process}
\end{figure}

\begin{enumerate}
    \item
        Setting the competition parameters, or “GAP parameters”, according to
        the competition site, the expected pilot level and the expected tasks.
        This happens once for each competition, at the outset, and must not be
        changed throughout the competition. See
        section~\ref{sec:use-of-tracklog-data}.
    \item
        Setting a task – this happens typically once per day on flyable days.
        See section~\ref{sec:task-setting}.
    \item
        Letting the pilots fly the task. See section~\ref{sec:flying-a-task}.
    \item
        Evaluating the task, by collecting all pilots’ track logs for this
        task, and determining for each pilot the distance flown and, if the end
        of speed section was reached, in what time this happened. See
        section~\ref{sec:task-evaluation}.
    \item
        Calculating the task validity based on the task’s statistical values
        such as fastest time to ESS, number of pilots in goal, average distance
        flown and several others. See section~\ref{sec:task-validity}.
    \item
        Points allocation: Calculating the maximum number of points awarded for
        distance, speed, leading and arrival, based on the task validity and
        the statistical values found in the task evaluation. See
        section~\ref{sec:points-allocation}.
    \item
        Scoring each pilot’s flight, by calculating the awarded points for
        distance, speed, leading and arrival. The outcome, the pilots’ total
        score, is the sum of these four values. See
        sections~\ref{sec:pilot-score} and ~\ref{sec:special-cases}.
    \item
        Ranking all pilots according to their total score for the task results.
        See section~\ref{sec:task-ranking}.
    \item
        Aggregation of task results for competition scoring and ranking. See
        section~\ref{sec:competition-ranking}.
\end{enumerate}
\end{document}
