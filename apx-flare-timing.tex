% !TEX TS-program = xelatex
% !TEX encoding = UTF-8 Unicode
\documentclass[gap.tex]{subfiles}
\begin{document}
The \href{https://github.com/BlockScope/flare-timing}{flare-timing}
implementation of GAP is a collection of apps, one for each step of the scoring
process, see Fig~\ref{fig:flare-timing}. Not being a single app that can store
intermediate results in memory, its workings and final results are written to
file with enough detail and evidence so that everything can be checked by hand.

\subsection{Scoring Sequence of Steps}

\begin{figure}[!ht]
    \centering
    \begin{tikzcd}
    \textit{comp}
    \texttt{.fsdb}
    \arrow[d, "\text{extract input}"] \\
    \textit{comp}
    \texttt{.comp-input.yaml}
    \arrow[d, "\text{cross zone}"] \\
    \textit{comp}
    \texttt{.cross-zone.yaml}
    \arrow[d, "\text{tag zone}"] \\
    \textit{comp}
    \texttt{.tag-zone.yaml}
    \arrow[d, "\text{align time}"] \\
    \texttt{.flare-timing/align-time/task-n/}
    \textit{pilot}
    \texttt{.csv}
    \arrow[d, "\text{discard further}"] \\
    \texttt{.flare-timing/discard-further/task-n/}
    \textit{pilot}
    \texttt{.csv}
    \arrow[d, "\text{mask track}"] \\
    \textit{comp}
    \texttt{.mask-track.yaml}
    \arrow[d, "\text{land out}"] \\
    \textit{comp}
    \texttt{.land-out.yaml}
    \arrow[d, "\text{gap point}"] \\
    \textit{comp}
    \texttt{.gap-point.yaml}
\end{tikzcd}

    \caption{Scoring as numbered steps from {\color{blue}original inputs} to
    {\color{csv}\texttt{*.csv}} or \texttt{*.yaml} outputs.}
    \label{fig:flare-timing}
\end{figure}

Starting with an \texttt{*.fsdb} comp and related \texttt{*.igc} or
\texttt{*.kml} track logs, the scoring process proceeds along this 
sequence of steps where outputs from earlier steps may be referred to later on;
\begin{enumerate}
    \item
        Extract the inputs with
        \href{https://github.com/BlockScope/flare-timing/tree/master/flare-timing/prod-apps/extract-input}{extract-input}.
    \item
        Trace the route of the shortest path to fly a task with
        \href{https://github.com/BlockScope/flare-timing/tree/master/flare-timing/prod-apps/task-length}{task-length}.
    \item
        Find pairs of fixes crossing over zones with
        \href{https://github.com/BlockScope/flare-timing/tree/master/flare-timing/prod-apps/cross-zone}{cross-zone}.
    \item
        Interpolate between crossing fixes for the time and place where a track
        tags a zone with
        \href{https://github.com/BlockScope/flare-timing/tree/master/flare-timing/prod-apps/tag-zone}{tag-zone}.
    \item
        Index fixes from the time of first crossing with
        \href{https://github.com/BlockScope/flare-timing/tree/master/flare-timing/prod-apps/align-time}{align-time}.
    \item
        Discard fixes that get further from goal and note leading area with
        \href{https://github.com/BlockScope/flare-timing/tree/master/flare-timing/prod-apps/discard-further}{discard-further}.
    \item
        Mask a task over its tracklogs with
        \href{https://github.com/BlockScope/flare-timing/tree/master/flare-timing/prod-apps/mask-track}{mask-track}.
    \item
        Group and count land outs with
        \href{https://github.com/BlockScope/flare-timing/tree/master/flare-timing/prod-apps/land-out}{land-out}.
    \item
        Score the competition with
        \href{https://github.com/BlockScope/flare-timing/tree/master/flare-timing/prod-apps/gap-point}{gap-point}.  
\end{enumerate}

\newpage
\subsection{Extracting Inputs}
In the \texttt{*.fsdb} FS keeps both inputs and outputs. We're only interested
in a subset of the input data, just enough to do the scoring;

\begin{description}
    \item[Competition] id, name, location, dates and UTC offset.
    \item[Nominal] launch, goal, time, distance and minimal distance.
    \item[Task] name and type of task, zones, speed section, start gates and pilots.
    \item[Zone] name, latitude, longitude and altitude and radius if a cylinder
    \item[Pilot] name and either absentee status or track log file name.
\end{description}

Something to be aware of when parsing the \texttt{XML} of the \texttt{*.fsdb}
is that attributes may be missing and in that case we'll have to infer the
defaults used by FS. This is done by looking at the source code of FS as there
is no schema for the \texttt{XML} that could also be used to set default
values.

\begin{lstlisting}[language=XML, caption={Overall *.fsdb structure, filtered for input.}]
<Fs>
  <FsCompetition id="7592" name="2012 Hang Gliding Pre-World Forbes" location="Forbes, Australia"
      from="2012-01-05" to="2012-01-14" utc_offset="11">
    <!-- Nominals are set once for a competition but beware, they are repeated per task. -->
    <FsScoreFormula min_dist="5" nom_dist="80" nom_time="2" nom_goal="0.2" />
    <FsParticipants>
      <FsParticipant id="23" name="Gerolf Heinrichs" />
      <FsParticipant id="106" name="Adam Parer" />
    </FsParticipants>
      <!-- Flags on how to score are also set for the competition but pick them up from the task. -->
      <FsTask name="Day 8" tracklog_folder="Tracklogs\day 8">
        <FsScoreFormula use_distance_points="1" use_time_points="1" use_departure_points="0" use_leading_points="1" use_arrival_position_points="1" use_arrival_time_points="0" />
        <FsTaskDefinition ss="2" es="5" goal="LINE" groundstart="0">
          <!-- Not shown here but each FsTurnpoint has open and close attributes. -->
          <FsTurnpoint id="FORBES" lat="-33.36137" lon="147.93207" radius="100" />
          <FsTurnpoint id="FORBES" lat="-33.36137" lon="147.93207" radius="10000" />
          <FsTurnpoint id="MARSDE" lat="-33.75343" lon="147.52865" radius="5000" />
          <FsTurnpoint id="YARRAB" lat="-33.12908" lon="147.57323" radius="400" />
          <FsTurnpoint id="DAY8GO" lat="-33.361" lon="147.9315" radius="400" />
          <!-- This was an elapsed time task so no start gates. -->
        </FsTaskDefinition>
        <FsTaskState stop_time="2012-01-14T17:22:00+11:00" />
        <FsParticipants>
          <!-- An empty participant element is a pilot absent from the task. -->
          <FsParticipant id="106" />
          <FsParticipant id="23">
            <FsFlightData tracklog_filename="Gerolf_Heinrichs.20120114-100859.6405.23.kml" />
          </FsParticipant>
        </FsParticipants>
      </FsTask>
    </FsTasks>
  </FsCompetition>
</Fs>
\end{lstlisting}

\newpage
\subsection{Finding an Optimal Route}

To find the best route \texttt{flare-timing} constructs a graph and finds the
shortest path connecting the nodes. It puts nodes on the turnpoint cylinder arc
boundaries and uses the haversine distance as the cost of connecting nodes in
the network. It would be expensive to construct and evaluate a large network
with the accuracy required so in an iterative process, as the arc of the circle
is closed, the density of nodes is increased. All happening on the FAI sphere,
this is the edge to edge distance.

\begin{lstlisting}[caption={Edge to edge distance of the optimal route, \texttt{edgeToEdge} node of \texttt{*.task-length.yaml}.}]
taskRoutes:
  edgeToEdge:
    distance: 159.373683
    legs:
    - 10.078208
    - 42.525217
    - 0
    - 64.949832
    - 41.820427
    legsSum:
    - 10.078208
    - 52.603424
    - 52.603424
    - 117.553256
    - 159.373683
    waypoints:
    - lat: -33.36047067
      lng: 147.93206999
    - lat: -33.43411056
      lng: 147.86878018
    - lat: -33.7159199
      lng: 147.55846831
    - lat: -33.7159199
      lng: 147.55846831
    - lat: -33.13199024
      lng: 147.57575486
    - lat: -33.35857718
      lng: 147.93468357
\end{lstlisting}

The naive way to measure task length would just be to connect the centers of each control zone. This is the point to point distance.

\begin{lstlisting}[caption={Which fixes are considered flown, \texttt{flying}
nodes of \texttt{*.task-length.yaml}.}]
taskRoutes:
  pointToPoint:
    distance: 169.10714
    legs:
    - 57.427511
    - 69.547668
    - 42.131961
    legsSum:
    - 57.427511
    - 126.975179
    - 169.10714
    waypoints:
    - lat: -33.36137
      lng: 147.93207
    - lat: -33.75343
      lng: 147.52864998
    - lat: -33.12908
      lng: 147.57322998
    - lat: -33.36099999
      lng: 147.93149998
\end{lstlisting}

Knowing that FS uses a plane to work out the shortest route in two dimensions,
on the the Universal Transverse Mercator projection.  We can also do that with
our graph algorithm.

\begin{lstlisting}[caption={Points on the plane, distances on the sphere, \texttt{projection/spherical} nodes of
\texttt{*.task-length.yaml}.}]
taskRoutes:
  projection:
    spherical:
      distance: 159.373683
      legs:
      - 10.078208
      - 42.525217
      - 0
      - 64.949832
      - 41.820427
      legsSum:
      - 10.078208
      - 52.603424
      - 52.603424
      - 117.553256
      - 159.373683
      waypoints:
      - lat: -33.36047067
        lng: 147.93206999
      - lat: -33.43411056
        lng: 147.86878018
      - lat: -33.7159199
        lng: 147.55846831
      - lat: -33.7159199
        lng: 147.55846831
      - lat: -33.13199024
        lng: 147.57575486
      - lat: -33.35857718
        lng: 147.93468357
\end{lstlisting}
\begin{lstlisting}[caption={Points and distances on the plane, \texttt{projection/planar} nodes of
\texttt{*.task-length.yaml}.}]
taskRoutes:
  projection:
    planar:
      distance: 159.144781
      legs:
      - 10.065441
      - 42.4942
      - 0
      - 64.761082
      - 41.820427
      legsSum:
      - 10.065441
      - 52.559642
      - 52.559642
      - 117.320723
      - 159.14115
      mappedPoints:
      - easting: 586715.834
        northing: 6308362.198
      - easting: 580759.282
        northing: 6300248.47
      - easting: 551744.701
        northing: 6269201.551
      - easting: 551744.701
        northing: 6269201.551
      - easting: 553704.761
        northing: 6333932.964
      - easting: 586960.882
        northing: 6308569.955
      mappedZones:
      - latZone: H
        lngZone: 55
\end{lstlisting}

\newpage
\subsection{Finding Zone Crossings}

First off, before we can determine if any zones have been crossed we'll have to
decide how to tell which parts of a track log are flown and which are walked or
driven in the retrieve car, possibly even back to goal.\footnote{Some pilots'
track logs will have initial values way off from the location of the device.
I suspect that the GPS logger is remembering the position it had when last
turned off, most likely at the end of yesterday's flight, somewhere near where
the pilot landed that day. Until the GPS receiver gets a satellite fix and can
compute the current position the stale, last known, position gets logged. This
means that a pilot may turn on their instrument inside the start circle but
their tracklog will start outside of it.}

To work out when a pilot is flying, select the longest run of fixes that are
not the same allowing for some stickiness when the GPS loses signal. For
example we might consider within ± 1m altitude or within ± 1/10,000th of
a degree of latitude or longitude to be in the same location and not recorded
during flight.

\begin{lstlisting}[caption={Which fixes are considered flown, \texttt{flying} nodes of \texttt{*.cross-zone.yaml}.}]
flying:
  - - Gerolf Heinrichs
    - loggedFixes: 4786
      flyingFixes:
      - 293
      - 4775
      loggedSeconds: 19140
      flyingSeconds:
      - 1172
      - 19100
      loggedTimes:
      - 2012-01-14T02:00:05Z
      - 2012-01-14T07:19:05Z
      flyingTimes:
      - 2012-01-14T02:19:37Z
      - 2012-01-14T07:18:25Z
\end{lstlisting}

Next we need to find and nominate every crossing of each control zone and from
among those work out which pair to select as the zone crossing. The same
control zone may be crossed multiple times and we need a sequence of crossings
ordered in time that fits the task\footnote{On a triangle course early fixes
may cross goal.}.

\begin{lstlisting}[caption={Selected crossings, one for each zone, \texttt{crossing} nodes of \texttt{*.cross-zone.yaml}.}]
crossing:
  - - Gerolf Heinrichs
    - zonesCrossSelected:
    - zonesCrossSelected:
      - crossingPair: ...
      - crossingPair: ...
      - crossingPair: ...
      - crossingPair: ...
      - crossingPair: ...
        - fix: 4711
          time: 2012-01-14T07:14:09Z
          lat: -33.35869799
          lng: 147.927904
        - fix: 4712
          time: 2012-01-14T07:14:13Z
          lat: -33.35934199
          lng: 147.928741
        inZone:
        - false
        - true
    - zonesCrossNominated: ...
\end{lstlisting}

\newpage
\subsection{Interpolating Zone Tagging}

Between the pair of fixes straddling a control zone, we need to interpolate the
point at which the pilot is most likely to have crossed and the time of this
tagging of the turnpoint\footnote{Currently \texttt{flare-timing} is just picking the fix inside the control zone.}.

\begin{lstlisting}[caption={Selected taggings, one for each zone, \texttt{tagging} nodes of \texttt{*.tag-zone.yaml}.}]
tagging:
  - - Gerolf Heinrichs
    - zonesTag:
      - fix: 294
        time: 2012-01-14T02:19:41Z
        lat: -33.36058598
        lng: 147.93161599
      - fix: 1367
        time: 2012-01-14T03:31:13Z
        lat: -33.41472398
        lng: 147.846
      - fix: 2208
        time: 2012-01-14T04:27:17Z
        lat: -33.708522
        lng: 147.530401
      - fix: 3915
        time: 2012-01-14T06:21:05Z
        lat: -33.13216898
        lng: 147.57301598
      - fix: 4712
        time: 2012-01-14T07:14:13Z
        lat: -33.35934198
        lng: 147.928741
\end{lstlisting}

\begin{lstlisting}[caption={The count of pilots tagging zones and the times of these taggings, \texttt{timing} nodes of \texttt{*.tag-zone.yaml}.}]
timing:
- zonesSum:
  - 76
  - 83
  - 75
  - 56
  - 29
  zonesFirst:
  - 2012-01-14T02:00:37Z
  - 2012-01-14T02:43:04Z
  - 2012-01-14T04:26:06Z
  - 2012-01-14T06:21:05Z
  - 2012-01-14T07:14:13Z
  zonesLast:
  - 2012-01-14T03:10:03Z
  - 2012-01-14T04:19:08Z
  - 2012-01-14T07:16:55Z
  - 2012-01-14T08:06:43Z
  - 2012-01-14T08:12:35Z
  zonesRankTime:
  - - ...
  - - 2012-01-14T07:14:13Z
    - 2012-01-14T07:30:57Z
    - 2012-01-14T07:35:05Z
    - ...
  zonesRankPilot:
  - - ...
  - - Gerolf Heinrichs
    - Attila Bertok
    - Jonas Lobitz
    - ...
\end{lstlisting}

\newpage
\subsection{Time Aligning Flights}

Next we align the tracks in time and work out the distance flown for each fix.

\begin{lstlisting}[caption={Fixes aligned in time with distance flown, rows of \texttt{*.align-time.csv}}]
leg,time,lat,lng,tickLead,tickRace,distance
0,2012-01-14T02:19:37Z,-33.36082199,147.93187399,-1407.0,-1407.0,159.275
1,2012-01-14T02:19:41Z,-33.36058599,147.93161599,-1403.0,-1403.0,159.279
...
1,2012-01-14T03:31:09Z,-33.41410199,147.84640799,2885.0,2885.0,149.643
2,2012-01-14T03:31:13Z,-33.41472399,147.84600000,2889.0,2889.0,149.565
...
2,2012-01-14T04:27:13Z,-33.70794300,147.53052900,6249.0,6249.0,106.253
3,2012-01-14T04:27:17Z,-33.70852200,147.53040100,6253.0,6253.0,106.066
...
3,2012-01-14T06:21:01Z,-33.13285599,147.57273699,13077.0,13077.0,42.015
4,2012-01-14T06:21:05Z,-33.13216899,147.57301599,13081.0,13081.0,41.938
...
4,2012-01-14T07:14:09Z,-33.35869799,147.92790400,16265.0,16265.0,0.421
5,2012-01-14T07:14:13Z,-33.35934198,147.92874100,16269.0,16269.0,0.0
\end{lstlisting}


\subsection{Incrementally Closer to Goal}

Then we cull any fixes that get further from goal and work out the leading area.

\begin{lstlisting}[caption={Fixes getting closer to goal with leading area, rows of \texttt{*.discard-further.csv}}]
leg,tickLead,tickRace,distance,area
0,-1407.0,-1407.0,159.275,0.0
1,-1367.0,-1367.0,159.264,0.0
...
1,2885.0,2885.0,149.643,0.0
2,2889.0,2889.0,149.565,1.4747201784719584e-9
...
2,6249.0,6249.0,106.253,1.2492010001497237e-9
3,6253.0,6253.0,106.066,5.430166826250327e-9
...
3,13077.0,13077.0,42.015,1.6116016630161737e-9
4,13081.0,13081.0,41.938,1.8495333543888993e-9
...
4,16265.0,16265.0,0.421,3.46272001957514e-11
5,16269.0,16269.0,0.0,6.306949125520327e-11
\end{lstlisting}

\subsection{Masking Task over Track}
\subsection{Assessing Difficulty}
\subsection{Collating Scores}

\end{document}
