\label{sec:task-validity}
The task validity is a value between 0 and 1 and measures how suitable
a competition task is to evaluate pilots’ skills. It is calculated for each
task after the task has been flown, by multiplying the three validity
coefficients: Launch validity, distance validity, and time validity.
\begin{equation*}
    TaskValidity = LaunchValidity * DistanceValidity * TimeValidity
\end{equation*}

\subsection{Launch Validity}
\label{sec:launch-validity}
Launch validity depends on nominal launch and the percentage of pilots actually
present at take-off who launched. If the percentage of pilots on take-off who
launch is equal to nominal launch, or higher, then launch validity is 1. If,
for example, only 20\% of the pilots present at take-off launch, launch
validity goes down to about 0.1.

The reasoning behind launch validity: Launch conditions may be dangerous, or
otherwise unfavourable.  If a significant number of pilots at launch think that
the day is not worth the risk of launching, then the gung-ho pilots who did go
will not get so many points. This is a safety mechanism.

‘Pilots present’ are pilots arriving on take-off, with their gear, with the
intention of flying. For scoring purposes, ‘Pilots present’ are all pilots not
in the ‘Absent’ status (ABS): Pilots who took off, plus pilots present who did
not fly (DNF). DNFs need to be attributed carefully. A pilot who does not
launch due to illness, for instance, is not a DNF, but an ABS.
\begin{align*}
    LVR &= \min(1, \frac{NumberOfPilotsFlying}{NumberOfPilotsPresent * NominalLaunch}) \\
    LaunchValidity &= 0.027 * LVR + 2.917 * LVR^2 - 1.944 * LVR^3
\end{align*}

\begin{figure}[h!]
    \centering
    \begin{tikzpicture}
\begin{axis}[ domain=0:1
            , xlabel=LVR
            , ylabel=Launch Validity
            , width=0.8\textwidth
            , xtick={ 0, 0.2, 0.4, 0.6, 0.8, 1.0 }
            , samples=25
            ]
    \addplot[blue] {0.027*x + 2.917*x^2 - 1.944*x^3};
\end{axis}
\end{tikzpicture}

    \caption{Launch validity curve}
\end{figure}

\subsection{Distance Validity}
\label{sec:distance-validity}
Distance validity depends on nominal distance, nominal goal, the longest
distance flown and the sum of all distances flown beyond minimum distance. If
the task distance is quite short in relation to nominal distance, the day is
probably not a good measure of pilot skill because there would not be many
decisions to make.

If a task is longer than nominal distance, the day will not be devalued because
of distance validity, even if the nominal goal parameter value is not achieved,
as long as a fair percentage of pilots fly a good distance. This sounds like
a vague statement, but the task setter should try to set tasks that are
reasonable for the day and achievable. If everyone lands in goal, you must ask
if this was a valid test of skill - it probably was if the fastest time and the
distance flown were reasonably long. If everyone lands short of goal, was it an
unsuitable task but still a good test of pilot skill? You also can have the
case where a task that is shorter than nominal distance, has a distance
validity of almost 1. This will happen when a large percentage of the pilots
fly a large percentage of the course but, in this case, you still have
a practical devaluation because there will be little spreading between pilots’
scores.

In the formula below, \(p\) denotes an individual pilot.
\begin{align*}
    SumOfFlownDistancesOverMinDist &= \sum_p \max(0, FlownDist_p - MinDist) \\
    NominalDistanceArea &= \frac{(a + b)}{2} \\
    a &= (NomGoal + 1) * (NomDist - MinDist) \\
    b &= \max(0, NomGoal * (BestDist - NomDist) \\
    DVR &= \frac{SumOfFlownDistancesOverMinDist}{NumberOfPilotsFlying * NominalDistanceArea} \\
    DistanceValidity &= \min(1, DVR)
\end{align*}

\subsection{Time Validity}
Time validity depends on the fastest time to complete the speed section, in
relation to nominal time. If the fastest time to complete the speed section is
longer than nominal time, then time validity is always equal to 1.

If the fastest time is quite short, the day is probably not a good measure of
pilot skill because there would not be many decisions to make and, because of
this, luck can distort scores as there will be little possibility to recover
any accidental loss of time.

If no pilot finishes the speed section, then time validity is not based on time
but on distance: The distance of the pilot who flies the furthest in relation
to nominal distance is then used to calculate the time validity the same way as
if it was the time.
\begin{align*}
    if \ one \ pilot \ reached \ ESS \\
    TVR &= \min(1, \frac{BestTime}{NominalTime}) \\
    if \ no \ pilot \ reached \ ESS \\
    TVR &= \min(1, \frac{BestDistance}{NominalDistance}) \\
    TimeValidity &= \max(0, \min(1, -0.271 + 2.912 * TVR - 2.098 * TVR^2 + 0.457 * TVR^3))
\end{align*}

\begin{figure}[h!]
    \begin{tikzpicture}
\begin{axis}[ domain=0:1
            , xlabel=Best/Nominal (Time or Distance)
            , ylabel=Time Validity
            , width=0.8\textwidth
            , xtick={ 0, 0.2, 0.4, 0.6, 0.8, 1.0 }
            , samples=100
            ]
    \addplot[blue] {max(0, min(1, -0.271 + 2.912*x - 2.098*x^2 + 0.457*x^3))};
\end{axis}
\end{tikzpicture}

    \caption{Time validity curve}
\end{figure}
