\label{sec:shortest-route}
Given the task control zones, how is the optimized task
distance~\ref{sec:task-distance} worked out? FS uses a method of line circle
intersections on a projected plane.

\subsection{Line-Circle Intersections on a Projected Plane}
Currently, this method mistreats the case where after ESS one or several
turnpoints have to be reached with centres different from ESS’ centre point:
The two legs to and from ESS are optimized for distance as well, which can
result in a longer last leg before ESS than what pilots experience in reality.
As a result, pilots can reach ESS with a shorter flown distance than what FS
indicates is the distance required to reach ESS. This only affects pilots
landing between ESS and goal, but does not represent a disadvantage to those
pilots. The speed section distance calculation used for scoring (see below) is
not affected by this.

Generally the algorithm does the following:

\begin{enumerate}
    \item
        Converts all coordinates to planar.  Planar coordinates are UTM Easting
        and Northing. The UTM Zone is automatically determined from the
        Longitude of the first point in the task definition. Then all other
        conversions are forced to be in the very same UTM Zone even if they are
        outside. This avoids anomalies when a competition is run in area
        covering two adjacent UTM Zones.
    \item
        Finds the shortest route in the projection plane, an adjustment made in
        three steps;
\begin{enumerate}
    \item
        Take the previous touching point \(A\).
    \item
        Take the next touching point \(B\).
    \item
        Calculate a point \(R\) on the circle so the distance
        $\overleftrightarrow{ARB}$ is shortest possible, considering three
        possible crossings by line of the circle;
\begin{enumerate}
    \item
        Line segment $\overleftrightarrow{AB}$ \textbf{crosses the circle in
        1 point}, see Fig~\ref{fig:planar-one-cross}.\\The point \(R\) is the
          intersection of $\overleftrightarrow{AB}$ and the circle.
\begin{figure}[ht]
    \centering
    \begin{tikzpicture}[scale=1.1]
\tkzDefPoint(1,1){C}
\tkzDefPoint(2,2){A}
\tkzDefPoint(5,0){B}
\tkzDefPoint(3,3){D}


\tkzInterLC(A,B)(C,D)
\tkzGetPoints{E}{R}

\tkzDrawCircle[color=blue, fill=blue!40, opacity=.5](C,D)

\tkzDrawPoints[color=blue, size=12](C,A,B)
\tkzDrawPoints[color=red, size=12](R)

\tkzDrawSegment(A,B)

\tkzLabelPoints[above left](A)
\tkzLabelPoints[below right](B)
\tkzLabelPoints[above right](R)
\tkzLabelPoints[below left](C)

\tkzLabelCircle[draw, fill=yellow!20, text centered](C,D)(135){turnpoint}
\end{tikzpicture}

    \caption{Line segment crosses the circle in 1 point}
    \label{fig:planar-one-cross}
\end{figure}

    \item
        Line segment $\overleftrightarrow{AB}$ \textbf{crosses the circle in
        2 points.}, see Fig~\ref{fig:planar-two-cross}.\\ The point \(R\) is
          the intersection of the circle and the line segment
          $\overleftrightarrow{AC'}$ where \(C'\) is the closest point on
          $\overleftrightarrow{AB}$ to the center of the circle.

\begin{figure}[ht]
    \centering
    \begin{tikzpicture}[scale=1.1]
\tkzDefPoint(1,1){C}
\tkzDefPoint(0,4){A}
\tkzDefPoint(5,0){B}
\tkzDefPoint(3,3){D}

\tkzDefPointBy[projection=onto A--B](C) \tkzGetPoint{C'}

\tkzInterLC(A,B)(C,D)
\tkzGetPoints{R}{E}

\tkzDrawCircle(C,D)

\tkzDrawPoints[color=blue, size=12](C,A,B)
\tkzDrawPoints[color=red, size=12](R,C')

\tkzDrawSegment(A,B)
\tkzDrawSegment[style=dashed](C,C')
\tkzMarkRightAngle(A,C',C)

\tkzLabelPoints[above left](A)
\tkzLabelPoints[below right](B)
\tkzLabelPoints[above right](R,C')
\tkzLabelPoints[below left](C)
\end{tikzpicture}

    \caption{Line segment crosses the circle in 2 points}
    \label{fig:planar-two-cross}
\end{figure}

    \item
        Line segment $\overleftrightarrow{AB}$ \textbf{doesn’t cross the
        circle}, see Fig~\ref{fig:planar-no-cross}.\\ The new point \(R\) is
        the intersection of the circle and line $\overleftrightarrow{CK}$ where
        \(K\) is point on $\overleftrightarrow{AB}$ so that \(|AK| / |BK|
        = |AT| / |BT|\) where \(T\) is current best touching point.  The sum
        \(|AR| + |BR|\) is at a minimum when angles $\angle{ARC}$ and
        $\angle{BRC}$ are equal. Angles are equal when \(|AK| / |BK| = |AR|
        / |BR|\).

\begin{figure}[ht]
    \centering
    \begin{tikzpicture}[scale=0.9]
\tkzDefPoint(1,1){C}
\tkzDefPoint(0,5){A}
\tkzDefPoint(6,6){B}
\tkzDefPoint(3,3){D}

\tkzDefBarycentricPoint(A=3,B=1)
\tkzGetPoint{K}

\tkzDefBarycentricPoint(A=6,B=1)
\tkzGetPoint{E}

\tkzInterLC(C,K)(C,D)
\tkzGetPoints{Q}{R}

\tkzInterLC(C,E)(C,D)
\tkzGetPoints{S}{T}

\tkzDrawCircle[color=blue, fill=blue!40, opacity=.5](C,D)

\tkzDrawPoints[color=blue, size=12](C,A,B)
\tkzDrawPoints[color=red, size=12](K,R,T)

\tkzDrawSegment(A,B)
\tkzDrawSegment[style=dashed](C,K)

\tkzLabelPoints[above right](R)
\tkzLabelPoints[above left](A,K,T)
\tkzLabelPoints[below right](B)
\tkzLabelPoints[below left](C)

\tkzLabelCircle[draw, fill=yellow!20, text centered](C,D)(135){turnpoint}
\end{tikzpicture}

    \caption{Line segment doesn't cross the circle}
    \label{fig:planar-no-cross}
\end{figure}

\end{enumerate}
\end{enumerate}

The algorithm is iterative, with every iteration it finds new route that is
(potentially) shorter than the one from previous iteration. When no further
shortening is possible the algorithm stops. For every Turnpoint in the route
is calculated best touching point. With every iteration the best touching
points are adjusted in the same order as the turnpoints in the task. Initially
the touching points are at the turnpoints and after 1st iteration they are at
the circles defining control zones.

After some iterations \(T\) and \(R\) become at the same place. The same is
valid if \(A\) and \(B\) are inside the circle.

If the current control zone is goal line then the best touching point is just
the point on the line closer to point \(A\) because the goal line can
be only at the end of the task, there is no point \(B\).

    \item
        Converts the coordinates of the shortest route back to geodetic.
Nothing special, just UTM to Latitude, Longitude conversion.
    \item
        Calculates the route distance according to geodetic coordinates.
Final distance calculation is made with the current FAI distance formula
between two points defined with geodetic coordinates.
\end{enumerate}

\end{document}
