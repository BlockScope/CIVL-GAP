Before the first task, the following parameters must be defined by the meet
director, or another person or group as defined by the competition’s local
regulations:
\begin{enumerate}
    \item Nominal Launch
    \item Nominal Distance
    \item Minimum Distance
    \item Nominal Goal
    \item Nominal Time
\end{enumerate}
The values set for these parameters define how each task’s validity is
calculated. They should therefore be chosen very carefully, considering the
realistic potential of the flying site. Setting the values too low will prevent
the formula from distinguishing between demanding, high quality tasks and
quick, easy low quality tasks which are sometimes the only option due to
weather conditions.

\begin{pg}
In paragliding competitions, the following must also be defined by the same
person or group who defines the first five competition parameters above:
\begin{enumerate}
    \setcounter{enumi}{5}
    \item Final glide decelerator
    \item Score-back time for stopped task
\end{enumerate}
\end{pg}

\subsection{Nominal Launch}
\label{sec:nominal-launch}
When pilots do not take off for safety reasons, to avoid difficult launch
conditions or bad conditions in the air, Launch Validity is reduced (see
section ~\ref{sec:launch-validity}). Nominal Launch defines a threshold as a percentage of the pilots
in a competition. Launch Validity is only reduced if fewer pilots than defined
by that threshold decide to launch.

The recommended default value for Nominal Launch is 96\%, which means that
Launch Validity will only be reduced if fewer than 96\% of the pilots present at
launch chose to launch.

\subsection{Nominal Distance}
Nominal distance should be set to the expected average task distance for the
competition. Depending on the other competition parameters and the distances
actually flown by pilots, tasks shorter than Nominal Distance will be devalued
in most cases. Tasks longer than nominal distance will usually not be devalued,
as long as the pilots fly most of the distance.

In order for GAP to be able to distinguish between good and not-so-good tasks,
and devalue the latter, it is important to set nominal distance high
enough\footnote{See also this excellent series of articles on the subject: Part
1: \url{http://ozreport.com/1360767307}; Part 2:
\url{http://ozreport.com/1360858575}; Part3:
\url{http://ozreport.com/1360944246}}.  

\subsection{Minimum Distance}
The minimum distance awarded to every pilot who takes off. It is the distance
below which it is pointless to measure a pilot's performance. The minimum
distance parameter is set so that pilots who are about to "bomb out" will not
be tempted to fly into the next field to get past a group of pilots – they all
receive the same amount of points anyway.

\subsection{Nominal Goal}
The percentage of pilots the meet director would wish to have in goal in
a well-chosen task. This is typically 20 to 40\%. This parameter has a very
marginal effect on distance validity (see section~\ref{sec:distance-validity}).

\subsection{Nominal Time}
Nominal time indicates the expected task duration, the amount of time required
to fly the speed section. If the fastest pilot’s time is below nominal time,
the task will be devalued. There is no devaluation if the fastest pilot’s time
is above nominal time.

Nominal time should be set to the expected “normal” task duration for the
competition site, and nominal distance / nominal time should be a bit higher
than typical average speeds for the area.

\begin{pg}
\subsection{Final Glide Decelerator}
\label{sec:final-glide-decelerator}
The concept of a final glide decelerator was introduced to counteract
a development in competition paraglider design which favoured stability at high
speeds over stability at trim speed. The two final glide decelerators available
are:
\begin{description}
    \item [Conical end of speed section (CESS)]
        Instead of a cylinder, the end of speed section is an inverted cone.
        Time stops for a pilot when they enter that cone. For details
        see~\ref{sec:define-CESS}.
    \item [Arrival altitude time bonus (AATB)]
        The time bonus is calculated based on each pilot’s altitude above goal
        when crossing the end of speed section cylinder. This bonus is then
        deducted from the pilot’s speed section time to determine the score
        time. See also~\ref{sec:time-for-speed-section}.
\end{description}
A meet director may choose to use no final glide decelerator, or use either of
the two outlined above.
\end{pg}

\begin{pg}
\subsection{Score-back Time}
\label{sec:score-back-time}
In a stopped task, this value defines the amount of time before the task stop
was announced that will not be considered for scoring. The default is
5 minutes, but depending on local meteorological circumstances, it may be set
to a longer period for a whole competition. See also section~\ref{sec:stop-task-time}.
\end{pg}
