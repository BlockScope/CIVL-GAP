All international sporting events organised wholly or partly under the rules of
the Fédération Aéronautique Internationale (FAI) Sporting Code
\footnote{FAI Statutes, ..........................................................................  Chapter 1, ...................para. 1.6}
are termed FAI International Sporting Events
\footnote{FAI Sporting Code, Gen. Section, ......................................... Chapter 4, ...................para 4.1.2}
. Under the FAI Statutes
\footnote{FAI Statutes, .......................................................................... Chapter 1, ...................para 1.8.1}
, FAI owns and controls all rights relating to FAI International Sporting
Events. FAI Members
\footnote{FAI Statutes, .......................................................................... Chapter 2, ...................para 2.1.1; 2.4.2; 2.5.2 and 2.7.2}
shall, within their national territories
\footnote{FAI By-Laws, ......................................................................... Chapter 1, ...................para 1.2.1}
, enforce FAI ownership of FAI International Sporting Events and require them
to be registered in the FAI Sporting Calendar
\footnote{FAI Statutes, .......................................................................... Chapter 2, ...................para 2.4.2.2.5}
.

An event organiser who wishes to exploit rights to any commercial activity at
such events shall seek prior agreement with FAI. The rights owned by FAI which
may, by agreement, be transferred to event organisers include, but are not
limited to advertising at or for FAI events, use of the event name or logo for
merchandising purposes and use of any sound, image, program and/or data,
whether recorded electronically or otherwise or transmitted in real time. This
includes specifically all rights to the use of any material, electronic or
other, including software, that forms part of any method or system for judging,
scoring, performance evaluation or information utilised in any FAI
International Sporting Event
\footnote{FAI By-Laws, ......................................................................... Chapter 1, ...................paras 1.2.2 to 1.2.5}
.

Each FAI Air Sport Commission
\footnote{FAI Statutes, .......................................................................... Chapter 5, ...................paras 5.1.1, 5.2, 5.2.3 and 5.2.3.3}
may negotiate agreements, with FAI Members or other entities authorised by the
appropriate FAI Member, for the transfer of all or parts of the rights to any
FAI International Sporting Event (except World Air Games events
\footnote{FAI Sporting Code, Gen. Section, ......................................... Chapter 4, ...................para 4.1.5}
) in the discipline
\footnote{FAI Sporting Code, Gen. Section, ......................................... Chapter 2, ...................para 2.2.}
, for which it is responsible
\footnote{FAI Statutes, .......................................................................... Chapter 5, ...................para 5.2.3.3.7}
or waive the rights. Any such agreement or waiver, after approval by the
appropriate Air Sport Commission President, shall be signed by FAI Officers
\footnote{FAI Statutes, .......................................................................... Chapter 6, ...................para 6.1.2.1.3}
.

Any person or legal entity that accepts responsibility for organising an FAI
Sporting Event, whether or not by written agreement, in doing so also accepts
the proprietary rights of FAI as stated above. Where no transfer of rights has
been agreed in writing, FAI shall retain all rights to the event. Regardless of
any agreement or transfer of rights, FAI shall have, free of charge for its own
archival and/or promotional use, full access to any sound and/or visual images
of any FAI Sporting Event. The FAI also reserves the right to arrange at its
own expense for any and all parts of any event to be recorded.

Editor’s note: Hang-gliding and paragliding are sports in which both men and
women participate. Throughout this document the words “he”, “him” or “his” are
intended to apply equally to either sex unless it is specifically stated
otherwise.
