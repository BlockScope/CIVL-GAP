\label{sec:task-setting}
\subsection{Definition of a task}
\label{sec:task-definition}
A task can be either a race task or an open distance task.

\subsubsection{Race task}
A race task definition consists of:
\begin{enumerate}
    \item A launch point, given as GPS coordinates
    \item A number of control zones
    \item A goal
    \item
        An indication which of the control zones is the start (start of speed
        section)
    \item
        If goal does not serve as end of speed section: An indication which of
        the control zones is the end of speed section, along with its specific
        parameters (such as incline and radius for CESS, altitude time bonus
        factor for AATB)
    \item A launch time window
    \item A start procedure, including timing
    \item Optionally, a task deadline
\end{enumerate}

\begin{pg}
In exceptional circumstances, with regard to restricted launch areas and poor
flying conditions, to ensure the task is fair for 2/3rds of the pilots, a task
may be run without leading/departure points. This is to be declared at the task
briefing.
\end{pg}

\subsubsection{Open distance task}
An open distance task definition consists of:
\begin{enumerate}
    \item A launch point, given as GPS coordinates
    \item A number of control zones
    \item Optionally, an indication which of the control zones is the start
    \item Optionally, a direction for the final, open distance leg
    \item A launch time window
    \item If a start control zone exists: A start time
    \item A task deadline
\end{enumerate}

\subsection{Definition of control zones}
Control zones are geographical areas which must be reached by the pilots in the
course of a task. The three types of control zones are the turnpoint cylinder,
the conical end of speed section and the goal semi-circle.

\subsubsection{Definition of a turnpoint cylinder}
A turnpoint cylinder is defined as:
\begin{enumerate}
    \item A centre point \(c\), given as GPS coordinates
    \item A radius \(r\), given in meters
    \item
        An indication whether the cylinder is an “exit” or an “enter” cylinder.
        This defines whether the corresponding turnpoint is considered reached
        by a pilot when crossing the cylinder’s boundary from its inside to the
        outside, or from its outside to the inside.
\end{enumerate}

A turnpoint cylinder is then given as the cylinder with radius \(r\) around the
axis which cuts the \(x/y\) plain orthogonally at the cylinder’s centre point
\(c\).  For task evaluation purposes, only the cylinder’s projection in the
\(x/y\) plain is considered: a circle of radius \(r\) around \(c\).

Note that for start cylinders (SSS), “enter” only makes sense if the following
turnpoint cylinder lies within the SSS cylinder. Likewise, an “exit” only makes
sense if the first turnpoint lies outside of the SSS cylinder. Currently, the
start direction cannot be set within FS. Instead the program automatically
scores according to this logic.

\subsubsection{Definition of conical end of speed section}
\label{sec:define-CESS}

\begin{figure}[h]
    \centering
    \begin{tikzpicture}[scale=1.4]
\tkzInit[xmin=-6,xmax=4,ymin=-1,ymax=3]
\tkzDefPoint(4,0){R}
\tkzDefPoint(0,2){C}
\tkzDefPoint(0,0){C'}
\tkzDefPoint(4,2){A}
\tkzDefPoint(-4,2){B}
\tkzDefPoint(6,3){A'}
\tkzDefPoint(-6,3){B'}
\tkzDrawVectors(C',A' C',B' C',R)
\tkzDrawVectors[dashed, color=red](C,B)
\tkzMarkAngle[fill=yellow, size=3.8cm, opacity=.5](R,C',A)
\tkzFillPolygon[red!30, opacity=0.25](A',B',C')
\tkzLabelAngle[pos=2.5](R,C',A){$\arctan(i)$}
\tkzLabelPoints[above, shift={(0,0.2)}](C)
\tkzLabelPoints[below](C')
\tkzDrawPoints[shape=cross out,size=24,color=red](C)
\tkzDrawPoints[color=blue](C')
\tkzShowTransformation[symmetry=center C, length=1, color=red](B)
\tkzLabelSegment[above](C,B){$r$}
\tkzLabelCircle[ draw
               , fill=yellow!20
               , text centered
               ](C',A)(15){conical end of speed section}
\end{tikzpicture}

    \caption{Conical end of speed section from the side}
\end{figure}

\begin{pg}
A conical end of speed section is defined as:
\begin{enumerate}
    \item A centre point \(c\), given as GPS coordinates and altitude
    \item
        An incline \(i\), given as a ratio of altitude/distance to the centre
        point
    \item
        A radius \(r\), indicating the size of the circle resulting from the
        intersection between the cone and a horizontal plane at goal altitude
\end{enumerate}

A conical end of speed section is given as the cone resulting from an axis of
inclination \(i\) through the centre point \(c'\), which is equal to point
\(c\), but has its altitude lowered by \(r/i\) metres compared to \(c\).  The
incline is chosen for each task. The default value is 1:3.5. Values suggested
for use are between
1:2.5 and 1:4.
\end{pg}
\subsection{Definition of goal}
\label{sec:goal-definition}
A goal can be
\begin{enumerate}
    \item A cylinder (enter or exit), see above, or
    \item A line, defined by:
    \begin{enumerate}
        \item A centre point \(c\), given as GPS coordinates
        \item A length \(l\), given in meters
    \end{enumerate}
\end{enumerate}

\begin{pg}
If CESS is used, goal must either be a line at the cone’s centre point,
a cylinder at the cone’s centre point with a radius smaller than radius r of
the cone definition, or be located at a point that is different from the cone’s
centre point.
\end{pg}

\subsubsection{Goal line}
\label{sec:goal-line}
\begin{figure}[h]
    \centering
    \begin{tikzpicture}
\tkzDefPoint(-4,-4){P}
\tkzDefPoint(-6,-6){Q}
\tkzDrawCircle[ color=blue
              , fill=blue!40
              , opacity=.5
              ](P,Q)
\tkzDefPoint(4,4){C}
\tkzDefPoint(2,6){A}
\tkzDefPointBy[rotation= center C angle 180](A)
\tkzGetPoint{B}
\tkzDrawArc[ color=red
           , fill=red!40
           , opacity=.5
           ](C,B)(A)
\tkzDrawLines[color=red, add = 0 and 0](C,A C,B)
\tkzDrawPoints(C,P)
\tkzDrawSegment(C,P)
\tkzLabelPoints[above right](C)
\tkzLabelPoints[below left](P)
\tkzLabelSegment[below, shift={(-0.5, -0.5)}](C,A){$l/2$}
\tkzLabelSegment[below, shift={(-0.5, -0.5)}](C,B){$l/2$}
\tkzMarkRightAngle(P,C,A)
\tkzLabelCircle[ draw
               , fill=yellow!20
               , text centered
               ](P,Q)(135){last turnpoint}
\tkzLabelCircle[ draw
               , fill=yellow!20
               , text centered
               ](C,A)(-135){goal semi-circle}
\end{tikzpicture}

    \caption{Goal line definition for paragliding}
\end{figure}

The goal control zone consists of the semi-circle with radius \(l/2\) behind
the goal line, when coming from the last turn point that is different from the
goal line centre. Entering that zone without prior crossing of the goal line is
equivalent to crossing the goal line.  Physical lines can be used in addition
to the official, virtual goal line as defined by GPS coordinates, to increase
attractiveness for spectators and media, and to increase visibility for pilots.
Physical lines must be at least 50m long and 1m wide, made of white material
and securely attached to the ground. The physical line must match as closely as
possible the corresponding virtual line as defined by the goal GPS coordinates
and the direction of the last task leg. It must not be laid out further from
the previous turn point than the goal GPS coordinates.

By default, the goal line length \(l\) is set to 400m.

\subsection{Start procedures}
Start procedures define how an individual pilot’s start time is determined.
A start can be either air- or ground-started, and it can be either a race to
goal or an elapsed time start.

\subsubsection{Air start}
For air-started tasks, the competitors are free to launch any time during the
launch window, and to fly about, regardless of any turnpoint or start
cylinders, up to their race start. Race start is defined as the crossing of the
start cylinder in the prescribed direction for the last time before continuing
to flying through the remainder of the task.

\subsubsection{Ground start}
In a ground-started task, the race starts with the pilots’ launch. Since
a launch can be difficult to detect from a GPS track, the task setters must set
a cylinder around the launch area as the first turnpoint. A pilot’s start is
registered when he exits this cylinder for the first time. In the case of
a race to goal task, the launch window open time is the same as the first (or
only) start gate time.

\subsubsection{Race to goal}
A race to goal start is defined by one or more so-called “start gates”. The
first – or only – start gate is given as a daytime. Subsequent start gates are
given as a time interval, along with the number of start gates.

\textit{Example 1: “We have a Race to Goal task, the start gate opens at 13:00”}

\textit{Example 2: “We have a Race to Goal task with 5 start gates from 13:30 at a 20
minutes interval.”} – the start gate times in this case are 13:30, 13:50, 14:10,
14:30, and 14.50.  Pilots are free to start any time after the first (or
single) start gate. A pilot’s start time is then defined as the time of the
last start gate after which he started flying the speed section of the task.

\textit{Example 3: Given the start gates from Example 2 above, pilot A, crossing the
start cylinder at 13:49:01, will be given a start time of 13:30. Any pilot
starting after 14:50 will be given a start time of 14:50.}

Starting before the first (or only) start gate is considered a failed start.
Some refer to this as “jumping the gun”. The two disciplines handle failed
starts differently, see section~\ref{sec:early-start}.

\subsubsection{Elapsed time}
An elapsed time start is defined by a single “start gate”, given as a daytime.
Pilots are free to start any time after this start gate. A pilot’s start time
is then defined as the time at which he started flying the speed section of the
task. Each pilot has therefore an individual start time.

\textit{Example 1: “We have an Elapsed Time task, the start gate opens at 12:30”}
– pilot A starting at 12:31:03 has a start time of 12:31:03, pilot B starting
at 15:48:28 has a start time of 15:48:28.

\subsection{Distances}
\subsubsection{Task distance}
\label{sec:task-distance}
Task distance is defined as the path of shortest distance from the start point
to goal that touches all turnpoint cylinders, see appendix~\ref{sec:shortest-route}.

\subsubsection{Speed section distance}
Speed section distance is defined as the path of shortest distance from start
of speed section to end of speed section that touches all turnpoint cylinders.
The method to calculate this distance is the same as for task distance.

