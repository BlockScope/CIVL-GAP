\label{sec:task-evaluation}
From each pilot’s track, task evaluation determines the distance this pilot
flew along the task, and the time this pilot took to fly the speed section.

\subsection{Reaching a control zone}
\subsubsection{Reaching a turnpoint cylinder}
\label{sec:reaching-a-turnpoint-cylinder}
A cylinder is considered “reached” by a pilot if that pilot’s track log shows
the pilot crossing out of the cylinder in the case of an exit cylinder, or into
the cylinder in case of an enter cylinder, by containing at least one track
point closer to the cylinder’s centre than the cylinder radius (enter) or
further away from the cylinder’s centre than the cylinder radius (exit). During
task evaluation, only the \(x/y\) coordinates are considered, and a point must
lie within (enter) or outside of (exit) the circle representing the turnpoint
cylinder in the \(x/y\) plain. This is determined by measuring the distance
between a track point and the turnpoint. This distance must be greater (exit)
or smaller (enter) than the cylinder’s radius.  To compensate for different
distance calculations and different earth models in use by today’s GPS devices
(FAI sphere vs. WGS84 ellipsoid), a 0.5\% tolerance is used for this
calculation. This had to be introduced so that a pilot reading the distance to
the next cylinder centre from his GPS device can rely on having reached the
turnpoint when the distance displayed by the instrument is smaller than the
defined turnpoint cylinder radius.

For enter cylinders, this means that a tracklog point that is closer to the
turnpoint than \(r*1.005\) is considered proof of reaching the turnpoint. For
exit cylinders, this means that a tracklog point that is further away from the
turnpoint than \(r*0.995\) is considered proof of reaching the turnpoint.
Starting January 1st, 2015 – provided all distance measurements are then based
on the WGS84 ellipsoid (see~\ref{sec:distance}) – the tolerance for turnpoint
cylinders in CIVL’s FAI Category 1 events will be reduced to 0.01\% of the
cylinder radius, with a minimum of 5 meter. Organisers of FAI Category 2 events
may continue using the existing 0.5\% tolerance, to accommodate pilots flying
with instruments which calculate distance based on the FAI sphere.

The time when a control zone was reached is determined by the time a so-called
“crossing” occurred. A crossing is defined as a pair of consecutive track
points, of which at least one lies inside the band determined by the
turnpoint’s centre, its radius and the tolerance value.
\begin{align*}
    tolerance &= 0.5\% | 0.01\% \\
    minTolerance &= 0m | 5m \\
    turpoint_i : innerRadius_i &= \min(radius_i * (1 - tolerance), radius_i - minTolerance) \\
    turpoint_i : outerRadius_i &= \max(radius_i * (1 + tolerance), radius_i + minTolerance) \\
\end{align*}
\begin{equation*}
    crossing_i = \exists_j (a \land b) \lor (c \land d)
\end{equation*}
\begin{align*}
    a = distance(center_i, trackpoint_j) &>= innerRadius_i \\
    b = distance(center_i, trackpoint_{j+1}) &<= outerRadius_i \\
    c = distance(center_i, trackpoint_j) &<= outerRadius_i \\
    d = distance(center_i, trackpoint_{j+1}) &>= innerRadius_i \\
\end{align*}
The time of a crossing depends on whether it actually cuts across the actual
cylinder, or whether both points lie within the tolerance band, but on the same
side of the actual cylinder.
\begin{align*}
    crossing . time &= trackpoint_{j+1} . time \ when \ A \\
    crossing . time &= trackpoint_j . time \ when \ B \\
    crossing . time &= interpolateTime(trackpoint_{j+1}, trackpoint_{j+1}) . time \ when \ C \\
\end{align*}
\begin{align*}
    A &= (a \land b) \lor (c \land d) \land turnpoint_i = ESS \\
    B &= (a \land b) \lor (c \land d) \land turnpoint_i \neq ESS \\
    C &= (a \land d) \lor (c \land b) \\
    \\
    a = distance(center_i, trackpoint_j) &< radius_i \\
    b = distance(center_i, trackpoint_{j+1}) &< radius_i \\
    c = distance(center_i, trackpoint_j) &> radius_i \\
    d = distance(center_i, trackpoint_{j+1}) &> radius_i \\
\end{align*}
The method used to interpolate the crossing time is buried in FS’ code and will
have to be documented at a later point.

Finally, given all n crossings for a turnpoint cylinder, sorted in ascending
order by their crossing time, the time when the cylinder was reached is
determined.
\begin{align*}
    turnpoint_i &= SSS : reachingTime_i = crossing_n . time \\
    turnpoint_i &\neq SSS : reachingTime_i = crossing_0 . time
\end{align*}
\subsection{Reaching a conical end of speed section}
\begin{pg}
A conical end of speed section is considered “reached” by a pilot if that
pilot’s track log contains at least one point where the required glide angle to
the cone’s centre point is equal to or smaller than the cone’s incline.
\begin{align*}
    crossing_i : \exists j : \\
    \frac{trackpoint_j . altitude - center_{CESS} . altitude}{center_{CESS}, trackpoint_{j}} \leq incline \\
    \land \frac{trackpoint_{j - 1} . altitude - center_{CESS} . altitude}{distance(center_{CESS}, trackpoint_{j - 1})} > incline \\
    \\
    crossing_i . time &= interpolateTime(trackpoint_{j + 1}, trackpoint_{j + 1})
\end{align*}
Given all n crossings for a CESS, sorted in ascending order by their crossing
time, the time when CESS was reached is determined.
\[ reachingTime_{CESS} = crossing_0 . time \]
If goal is at the CESS centre point, and a pilot reaches goal without
previously entering the CESS, he is considered having reached CESS at the time
when he crossed the goal line.
\end{pg}

\subsection{Reaching goal}
\subsubsection{Goal cylinder}
Verification of a pilot reaching a goal cylinder is achieved by the same method
as verification of reaching a turnpoint cylinder (~\ref{sec:reaching-a-turnpoint-cylinder})

\subsubsection{Goal line}
To reach goal in the case of a goal line, the goal line must be crossed in
flight. This is achieved when a line drawn between two adjacent points in the
pilot’s tracklog crosses the goal line in the correct direction.

\begin{pg}
Entering the goal control zone (semi-circle behind the goal line,
see~\ref{sec:goal-line}) from any direction without prior crossing of the goal
line is equivalent to crossing the goal line.
\end{pg}

If a physical line is used, crossing either the virtual or the physical goal
line counts as having reached goal. An official observation (through a goal
marshal or similar) of a pilot crossing the line in flight overrules a negative
goal crossing decision based on the pilot’s tracklog. Not crossing a physical
goal line for obvious safety reasons must be considered in the pilots’ favour.

\begin{hg}
The physical goal line is crossed when the hang glider’s nose cuts the line, in
the correct direction, before a landing is made.
\end{hg}

\begin{pg}
The physical goal line is crossed when the paraglider pilot’s leading foot cuts
the line, in the correct direction, before a landing is made.
\end{pg}

\subsection{Flown distance}

\subsubsection{Race task}
\label{sec:race-task}
To determine a pilot’s flown distance, a first step determines which turnpoints
he reached considering all timing restrictions: launch within launch time
window, valid start, but only until the task deadline time. After the last
turnpoint the pilot reached, for every remaining track point, the shortest
distance to goal is calculated using the method described in section
\ref{sec:task-distance}. The flown distance is then calculated as task distance
minus the shortest distance the pilot still had to fly. Therefore, for scoring,
the pilot’s best distance along the course line is considered, regardless of
where the pilot landed in the end.

If a pilot flies less than minimum distance, he will be scored for this minimum
distance. This also applies to pilots who are not able to produce a valid GPS
tracklog, but for whom launch officials verify launch within the launch window.

If a pilot reaches goal, he will be scored for the task distance.
\begin{align*}
    \forall p : p \in PilotsLandingBeforeGoal : bestDistance_p &= \max(minimumDistance, taskDistance - shortest) \\
    shortest &= \min(\forall track_p . point_i shortestDistanceToGoal(track_p . point_i))) \\
    \forall p : p \in PilotsReachingGoal : bestDistance_p &= taskDistance
\end{align*}

\subsubsection{Open distance task}
In an open distance task, if a pilot lands before the last of any given control
zones, his flown distance is calculated according to \ref{sec:race-task}. For
pilots flying further than the last control zone, the flown distance is
calculated by adding the pilot’s best distance flown in the open distance part
of the task (after the last control zone) to task distance between launch and
the last control zone. If the task gave a direction for the open distance leg,
then the pilot’s best distance projected onto that direction is considered.

\subsection{Time for speed section}
\label{sec:time-for-speed-section}
The time a pilot took to fly the speed section is determined by his start time
(which is influenced by the task’s start procedure and the time he crossed the
start of speed section cylinder) and the time when he crossed the end of speed
section after reaching all previous turn points. The smallest unit for time
measurement is one second.

Pilots who do not reach the end of speed section cylinder do not get a time.
\begin{align*}
    \forall p : p \in PilotsReachingESS : time_p &= timeAtESS_p - startTime_p
\end{align*}

\begin{pg}
If AATB is used, pilots receive a time bonus for their altitude above goal when crossing the ESS cylinder.
The number of seconds they receive for each meter is expressed by the ATTB factor.

By default, the ATTB factor is \(0.45 \ s/m\). Values suggested for use are between \(0.2 \ s/m\) and \(0.7 \ s/m\).
\begin{align*}
    \forall p : p \in PilotsReachingESS \\
    arrivalAltitudeTimeBonus_p &= (crossing_{p.ESS} . altitude - goal . altitude) * ATTBfactor \\
    time_p &= timeAtESS_p - startTime_p - arrivalAltitudeTimeBonus_p
\end{align*}
\end{pg}
