% !TEX TS-program = xelatex
% !TEX encoding = UTF-8 Unicode
\documentclass[gap.tex]{subfiles}
\begin{document}
\label{sec:use-of-tracklog-data}
\subsection{Position}
Coordinates of positions, such as turn points or pilot positions, are always
given as WGS84 coordinates, based on the WGS84 ellipsoid. The coordinate format
is UTM by default, but other formats can be chosen by organisers as
appropriate.

\subsection{Distance}
\label{sec:distance}
In general, task evaluation occurs in the \(x/y\) plain, therefore distance
measurements are always exclusively horizontal measurements on the surface of
the Earth model.

\subsubsection{Earth Model}
\begin{pg}
    For paragliding the Earth model used is the WGS 84 ellipsoid.
\end{pg}

\begin{hg}
    For hang gliding the Earth model used is the FAI sphere \footnote{With the
    intent to change to WGS 84 ellipsoid in 2019.} \footnote{With a radius of
    6371.0 km}.  On a sphere distance between two points, identified by their
    radian coordinates \(lat_1/long_1\) and \(lat_2/long_2\), is calculated
    using the Haversine formula.
    \begin{align*}
        distLat &= lat_2 - lat_1 \\
        distLong &= long_2 - long_1 \\
        a &= \sin(\frac{distLat}{2})^2 + \cos lat_1 * \cos lat_2 * \sin(\frac{distLong}{2})^2 \\
        radianDistance &= 2 * \atantwo(\sqrt a, \sqrt{1 - a}) \\
        distance &= radianDistance * 6371000 \\
    \end{align*}
    To reproduce this formula in Excel, the following modification is necessary due
    to a different implementation of the \(arctan2\) function:
    \[ radianDistance = pi - 2 * \atantwo(\sqrt a, \sqrt{1 - a}) \]
\end{hg}

\subsubsection{Bonus Distance}
Any bonus given as a distance does not affect distance calculations between two
geographic points.

\begin{pg}
In paragliding, for final glide decelerators (\ref{sec:final-glide-decelerator})
and altitude bonus in stopped tasks (\ref{sec:distance-stopped-tasks}), altitude
is also considered.
\end{pg}

\begin{hg}
In hang gliding, for altitude bonus in stopped tasks (\ref{sec:distance-stopped-tasks}),
altitude is also considered.
\end{hg}

\subsection{Altitude}
All altitude evaluation is primarily based on barometric altitude, as recorded
in the flight instrument tracklog (the International Standard Atmosphere
pressure altitude QNE) and then when necessary corrected by the scoring
software for the pressure conditions of the task (QNH). GNSS altitude may be
taken into consideration (from the primary tracklog or a backup log) only in
case of problems with barometric logging.

Category 2 Organisers may choose to use the less accurate GPS altitude instead
of barometric altitude.

\subsection{Time}
Time evaluation is based on GPS time, as given in GPS tracklogs. For better
readability, times of the day may be expressed in local time for the
competition location.
\end{document}
