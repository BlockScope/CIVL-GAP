% !TEX TS-program = xelatex
% !TEX encoding = UTF-8 Unicode
\documentclass{article}
\usepackage{gap-civl}
\usepackage{subfiles}

\begin{document}
\label{sec:jtg}
\subsection{Penalty Ordering}


\begin{quotation}
Clarification of order in which penalties are applied
We have three categories of penalties in hang-gliding competitions, and two
categories in paragliding competitions. We specify in which order they are
applied.

3.3 12.4 Penalties
3.3.1 Status quo
 	These penalties are completely independent of any “Jump the Gun”-Penalty a pilot may have incurred
3.3.2 Proposal
     If a pilot incurs multiple penalties or bonuses, these are applied in the following order:
          1. “Jump the Gun”-Penalty
          2. Percentage penalty or bonus
          3. Absolute points penalty or bonus
     3.3.3 Reasoning
     Specify exactly the order in which penalties are applied.
\end{quotation}

Arguments

Can we please not do this? I don't want to see the idiosyncracies of FS
leaking into the rules.

What are the categories of penalties?

There's percentage (fractional) and point (absolute) penalties and then
there's something else related to jumping the gun.

Sticking with just fractional and point types for now, it is common sense in
which order these should be applied. We want to apply the fractional ones
first and then the absolute ones. The fractional ones are never going to
change the point ones but if we were to apply the points ones first then we'd
be taking a fraction of a subtotal and so the points demerited or
awarded would be changed.

Let's look how we can combine them if we have multiple penalties. If we take
the fractional ones, we can multiply them together before applying their
fractional product to the total. This gets us the same result as applying the
second one to the subtotal after multiplying the first with the total and so
on. Absolute penalties we can sum together before applying. This gets us the
same result as adding the second one to the subtotal after adding the first
to the total and so on.

On to jump the gun. The discipline of hang gliding has some tolerance of this
but paragliding has none. Start early in paragliding and you will be scored
for the distance between the launch point and the start. Start too early in
hang gliding and you will be scored for minimum distance. So both disciplines
will give you a score that is neither a fractional or an absolute change in
your total points. Your score is reset to some low value equivalent to some
poor pilot not getting very far along the course. A tough penalty indeed.

The way this is implemented in FS is a bit of a kludge but fits with one
reading of the rules. Unfortunately this convenience of implementation has
some unwanted side effects. Done this way, it is not a penalty that acts only
on the pilot the penalty is intended for. True, it takes away points from the
pilot starting too early but it also shifts the distribution of points to
other pilots. It's as if FS feeds a truncated track for this pilot through
the system, not their actual track. Any interactions relative to other pilots
are lost. If this pilot lead all the way, was the first to arrive at goal and
was the fastest then top points in these areas that would otherwise have been
awarded are shifted to the next in line. The pilot most closely following the
leader will now get the top leading points, the second to arrive will get the
top arrival points and the second fastest pilot will get the top time points.

This setup invites a strategy for one pilot to take all the risks but confer
top points to the next pilot, their team mate, if by starting early they can
get ahead and drag a second around the course. I can't say how practical this
is in practice, not having flown at the level of competition where this might
come into play.

What I can say as a software developer, is that testing this setup is harder
than going with a reset penalty. With it, we cannot test the algebra of
penalties in isolation, without also bringing along all the other parts of
scoring that calculate and assign points that then get doled out to the other
pilots.

If as an implementation detail we introduce a reset penalty then scoring
proceeds as if there is no such thing as jumping the gun. Those that started
too early get their points assigned in full but then reset, a very simple
calculation, a swap, a replacement. No points are shifted among the remaining
pilots. This is dead simple. The application of penalties and the interaction
of the fractional, absolute and reset penalties can be tested in isolation
from the other parts of the scoring implementation.

Adding resets as another type of penalty also fits with the reading of the
rules. A reset penalty should be applied last in order, after fractional and
absolute penalties have been applied. Where a reset is intended as a demerit,
intended to reduce a pilot's score then the reset cannot lift a pilot's score
back to the reset level in the situation where other penalties have already
lowered it below the reset level.

In hang gliding if someone jumps the gun by only a little, within a tolerance
window, then the early time is converted to an absolute penalty. These
accumulate at a rate of points per second and can then be treated like any
other absolute penalty.

Back to the proposal to apply penalties in order of jump-the-gun (reset or
absolute), fractional and then absolute. I would rather the order is
fractional, absolute (including absolute from jumping the gun) and then
reset (from jumping the gun).

\end{document}
