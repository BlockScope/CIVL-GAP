% !TEX TS-program = xelatex
% !TEX encoding = UTF-8 Unicode
\documentclass{article}
\usepackage{civl-proposal}

\title{Setting the Bar for Software Scoring}
\date{December 3, 2018}
\author{A proposal for the CIVL Plenary Meeting, Jan 31 - Feb 3, 2019\\\\
Phil de Joux\thanks{phil.dejoux@blockscope.com}}

\begin{document}
\maketitle
\section*{Problem}
I do not know by what criteria CIVL will assess new scoring software in order
to approve it.

\section*{Solution}
I propose that CIVL set the bar for approving software used for scoring with
GAP.

I suggest the following list of requirements:\\
\begin{enumerate}
    \item The program can read a common input format and produce a common
    output format, currently \texttt{*.fsdb} + \texttt{*.kml} + \texttt{*.igc}
    in and \texttt{*.html} out.
    \item The code that does the scoring is open source. Other parts of the
    application can be closed.
    \item There is somewhere to report and review issues with the code, eg.
    github issues.
    \item There is a way to contribute fixes, eg. pull requests.
    \item The scoring works on windows, linux and osx.
    \item There are build instructions that work on windows, linux and osx.
    \item There is documentation that explains the implementation.
    \item There are releases with release notes and change logs and the source
    is tagged at release points.
    \item The code that does the scoring can be run in isolation, apart from
    the application as a whole\footnote{The application as a whole would
    include any forms to collect competition, pilot and task inputs and
    displays such as flight statistics and maps whereas the code that does the
    scoring is just the code that implements GAP and any other code that
    enables these modules to read the common input files and produce the common
    output file and workings.}, in batch mode from the command line.
    \item Working steps can be written to a file format that is human readable.
    \item There are tests.
    \item The build and tests are run on a public continuous integration
    service\footnote{Examples of continuous integration services include
    \url{https://travis-ci.org/} and \url{https://circleci.com/}}.
\end{enumerate}

\section*{Discussion}
Points 1., 9. and 10. allow for a way to run the scoring program from the
command line (headless), reading common inputs, writing a common output and
allowing for scrutiny of the output and workings by humans.

Point 2. software that is open source allows for more scrutiny and is fairer
to pilots as nothing is hidden.

Points 5. \& 6. makes an expectation that the software builds and runs on the
three most common operating systems around today.

Points 3., 4., 8., 11. and 12. are needed for onboarding and working with
contributors and scrutineers.

We cannot yet produce a set of automated tests that allow CIVL to pass or fail
an implementation of GAP but I think that is a worthy and reasonable goal.
\end{document}
