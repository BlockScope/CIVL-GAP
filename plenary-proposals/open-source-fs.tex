% !TEX TS-program = xelatex
% !TEX encoding = UTF-8 Unicode
\documentclass{article}
\usepackage{civl-proposal}

\title{Make FS Public}
\date{December 3, 2018}
\author{A proposal for the CIVL Plenary Meeting, Jan 31 - Feb 3, 2019\\\\
Phil de Joux\thanks{phil.dejoux@blockscope.com}}

\begin{document}
\maketitle
\section*{Problem}
FS is closed source software and therefore not open to pilot scrutiny.

\section*{Solution}
I propose that CIVL make the \fref{https://github.com/FAI-CIVL/FS}{FS
repository} public after adding a license of our choosing. If a license is not
chosen two weeks after the plenary then we'll use the MPL-2.0 license.

\section*{Discussion}
FS as open source:\\
\begin{itemize}
    \item Would allow other projects to reuse snippets of code from FS.
    \item May attract more contributors.
    \item Allows for more scrutiny\footnote{Only members of the FAI-CIVL
    organisation on github can view the source code currently as the FS
    repository there is private.}.
    \item Is fairer to pilots as nothing is hidden.
    \item Raises the bar for scoring software seeking CIVL
    approval\footnote{How could we reasonably ask another implementer of GAP to
    open their code if the code for FS, CIVL's own implementation, is not
    open?}.
    \item Retains CIVL's ability to control distribution of the
    installer\footnote{Currently the one place to download the installer is
    from \url{http://fs.fai.org/}. If CIVL made their source code repository at
    \url{https://github.com/FAI-CIVL/FS} public, then the installer could also
    be attached to each release when published. It is an inbuilt security
    feature of github that only certain authorised users can cut a release. For
    more on this see
    \url{https://help.github.com/articles/repository-permission-levels-for-an-organization/}.}.
\end{itemize}
\end{document}
