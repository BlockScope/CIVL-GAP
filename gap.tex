% !TEX TS-program = xelatex
% !TEX encoding = UTF-8 Unicode
\documentclass{article}
\usepackage{civl}

% SEE: https://ctan.org/pkg/titlepages
\newcommand*{\titleGap}{\begingroup
\raggedleft
\vspace*{16\baselineskip}

\begin{tabularx}{\textwidth}{@{}rXr@{}}
    \textit{Fédération} & & \\
    \textit{Aéronautique} & & \multirow{2}{*}{} \\ \cline{2-3}
    \textit{Internationale} & & \\
\end{tabularx}

{\Huge CIVL GAP}\\
\vspace*{2\baselineskip}
{\Large Centralised Cross-Country Competition Scoring for}\\
{\Large Hang-Gliding and Paragliding}\\
\vspace*{2\baselineskip}
2016 Edition\\
Revision 1.1\\
Published February 26, 2017\\
\vspace*{10\baselineskip}

\begin{tabularx}{\textwidth}{@{}rXr@{}}
    \textit{Maison du Sport International} & & \\
    \textit{Av. de Rhodanie 54} & & \\
    \textit{CH-1007 Lausanne} & & \\
    \textit{(Switzerland)} & & \\
    \textit{Tél. +41 (0)21 345 10 70} & & \\
    \textit{Fax +41 (0)21 345 10 77} & & \\
    \textit{E-mail: \href{mailto:sec@fai.org}{sec@fai.org}} & & \\
    \textit{Web: \href{www.fai.org}{www.fai.org}} & & \\
\end{tabularx}

\endgroup}

\import{}{footer}
\pagestyle{gen}

\begin{document}
\import{}{header}

% SEE: http://tex.stackexchange.com/questions/86249/maketitle-text-before-title
{\let\newpage\relax\titleGap}\thispagestyle{pageone}

\newpage
Editor's note: Hang-gliding and paragliding are sports in which both men and women participate. Throughout this document the words "he", "him" or "his" are intended to apply equally to either sex unless it is specifically stated otherwise.\\

\begin{tabularx}{\textwidth}{|l|l|X|}
 \hline
    \textbf{Revision} & \textbf{Author} & \textbf{Changes} \\
 \hline
 2014 R1.1 & Joerg Ewald & Correction of a number of typos \\
 \hline
 2015 R1.0 & Joerg Ewald & Added changes from the 2015 CIVL Plenary:
    \begin{itemize}
        \item Use same leading coefficient calculation for both hang gliding and paragliding
        \item Use QNH for altitude measurement
        \item Final glide decelerators in paragliding no longer mandatory
        \item Distance measurement based on WGS84 ellipsoid postponed
    \end{itemize} \\
 \hline
 2016 R1.0 & Joerg Ewald & Added changes from the 2016 CIVL Plenary:
    \begin{itemize}
        \item PG: Double the amount of available leading points (reducing time points accordingly)
        \item PG: If no pilot in goal, the leading points weight is proportional to the task distance covered by pilots, up to 100 points.
        \item HG: Jump the gun penalty is now one point for every 2 seconds (was for every 3 seconds).
    \end{itemize} \\
 \hline
 2014 R1.1 & Joerg Ewald & Correct Figures 6 and 7 (point allocation graphs) \\
\hline
\end{tabularx}

\vspace*{\fill}
\section*{FEDERATION AERONAUTIQUE INTERNATIONALE}
\textbf{Maison du Sport International – Avenue de Rhodanie 54 – CH-1007 Lausanne – Switzerland}\\
Copyright 2014\\
All rights reserved. Copyright in this document is owned by the Fédération Aéronautique Internationale (FAI). Any person acting on behalf of the FAI or one of its Members is hereby authorised to copy, print, and distribute this document, subject to the following conditions:\\
\textbf{
    \begin{enumerate}
        \item The document may be used for information only and may not be exploited for commercial purposes.\\
        \item Any copy of this document or portion thereof must include this copyright notice.
    \end{enumerate}
}

\newpage
Note that any product, process or technology described in the document may be the subject of other Intellectual Property rights reserved by the Fédération Aéronautique Internationale or other entities and is not licensed hereunder.

\newpage
\tableofcontents

\newpage
\section{Introduction}
\subsection{Scope}
\subsection{Sources}
\subsection{Changes from previous edition}
\subsection{Differences between Hang-Gliding and Paragliding}

\newpage
\section{The GAP Philosophy}
\subsection{History}
\subsection{Scoring Process}

\newpage
\section{Definitions}
\subsection{Flights}
\subsection{Locations and distances}
\subsection{Times}

\newpage
\section{Use of Tracklog Data}
\subsection{Position}
\subsection{Distance}

In general, task evaluation occurs in the \(x/y\) plain, therefore distance
measurements are always exclusively horizontal measurements. The earth model used
is the FAI sphere, with a radius of 6371.0 km.

In paragliding, for final glide decelerators (5.6) and altitude bonus in stopped
tasks (12.3.6), altitude is also considered, but this does not affect distance
calculations between two geographic points.

In hang gliding, for altitude bonus in stopped tasks (12.3.6), altitude is also
considered, but this does not affect distance calculations between two geographic
points.

The distance between two points, identified by their radian coordinates
\(lat_1/long_1\) and \(lat_2/long_2\), is calculated using the Haversine formula.

\begin{eqnarray*}
    distLat = lat_2 - lat_1 \\
    distLong = long_2 - long_1 \\
    a = \sin(\frac{distLat}{2})^2 + \cos lat_1 * \cos lat_2 * \sin(\frac{distLong}{2})^2 \\
    radianDistance = 2 * \atantwo(\sqrt a, \sqrt{1 - a}) \\
    distance = radianDistance * 6371000 \\
\end{eqnarray*}

To reproduce this formula in Excel, the following modification is necessary due to a different
implementation of the \(arctan2\) function:

\[ radianDistance = pi - 2 * \atantwo(\sqrt a, \sqrt{1 - a}) \]

\subsection{Altitude}
\subsection{Time}

\newpage
\section{Competition Parameters}
\subsection{Nominal Launch}
\subsection{Nominal Distance}
\subsection{Minimum Distance}
\subsection{Nominal Goal}
\subsection{Nominal Time}
\subsection{Final Glide Decelerator}
\subsection{Score-back Time}

\newpage
\section{Task Setting}
\subsection{Definition of a task}
\subsubsection{Race task}
\subsubsection{Open distance task}
\subsection{Definition of control zones}
\subsubsection{Definition of a turnpoint cylinder}
\subsubsection{Definition of conical end of speed section}
\subsection{Definiton of goal}
\subsubsection{Goal line}
\subsection{Start procedures}
\subsubsection{Air start}
\subsubsection{Ground start}
\subsubsection{Race to goal}
\subsubsection{Elapsed time}
\subsection{Distances}
\subsubsection{Task distance}
\subsubsection{Speed section distance}

\newpage
\section{Flying a task}
\subsection{Race task}
\subsection{Open distance task}

\newpage
\section{Task evaluation}
\subsection{Reaching a control zone}
\subsubsection{Reaching a turnpoint cylinder}
\subsection{Reaching a conical end of speed section}
\subsection{Reaching goal}
\subsubsection{Goal cylinder}
\subsubsection{Goal line}
\subsection{Flown task}
\subsubsection{Race task}
\subsubsection{Open distance task}
\subsection{Time for speed section}

\newpage
\section{Task Validity}
\subsection{Launch Validity}
\subsection{Distance Validity}
\subsection{Time Validity}

\newpage
\section{Points Allocation}

\newpage
\section{Pilot score}
\subsection{Distance points}
\subsubsection{Difficulty calculation}
\subsubsection{Example for difficulty calculation}
\subsection{Time points}
\subsection{Leading points}
\subsubsection{Leading coefficient}
\subsubsection{Example}
\subsection{Arrival points}

\newpage
\section{Special cases}
\subsection{ESS but not goal}
\subsection{Early start}
\subsection{Stopped tasks}
\subsubsection{Stop task time}
\subsubsection{Requirements to score a stopped task}
\subsubsection{Stopped task validity}
\subsubsection{Scored time window}
\subsubsection{Time points for pilots at or after ESS}
\subsubsection{Distance points with altitude bonus}
\subsection{Penalties}

\newpage
\section{Task ranking}
\subsection{Overall task ranking}
\subsection{Female task ranking}
\subsection{Nation task ranking}

\newpage
\section{Competition ranking}
\subsection{Overall competition ranking}
\subsection{Female competition ranking}
\subsection{Nation competition ranking}
\subsection{Ties}

\newpage
\section{FTV - Fixed Total Validity}

\end{document}
